\newacronym{ctan}{CTAN}{Comprehensive TeX Archive Network}
\newacronym{faq}{FAQ}{Frequently Asked Questions}
\newacronym{pdf}{PDF}{Portable Document Format}
\newacronym{svn}{SVN}{Subversion}
\newacronym{wysiwyg}{WYSIWYG}{What You See Is What You Get}

\newglossaryentry{texteditor}
{
	name={editor},
	description={A text editor is a type of program used for editing plain text files.}
}

\chapter{Abstract}


This is where the first chapter begins...


\chapter{Related Publications}


This is where the first chapter begins...

\chapter{Introduction}


This is where the first chapter begins...

From here it is important to show how the multiple elements of the thesis blend together.

Multiple zoom levels, multiple temporal levels

Towards a dynamic and interactive model of a cell (E.Coli)



New mesocale models are being developed to better understand the pysiology of viruses and cells.
Right now only static models exists, but thanks to our technology we are paving the way toward interactive visualization of larger dynamic structures.



%Understanding how we work means understanding a cascade of events from the microscopic to macroscopic levels.
%
%To begin being with this ambitious enterprise it is necessary to starting studying how processes work on the atomic level, the smallest level which we  understand to this day.

\section{Foreword}

\textbf{Importance of simulation if biology}

In-silico modeling in the domain of biology have significantly increased over the last decades.

Development of new software and better hardware now allow to run complex simulations and analysis that provide valuable insight to biologists.

In 2013 Martin Karplus, Michael Levitt and Arieh Warshel have been awarded the Nobel prize in Chemistry for their work of theoretical modelling for complex chemical system.

Their work showcase the importance of theoretical modelling as a tool to complement experimental techniques.

Indeed the collected information is served to formulate new hypothesis which can be later on verified in wet laboratories.

Wet-lab experiments are usually expensive to conduct and therefore the use of computer simulation if very relevant because it may help researchers to save the time and money needed to run such experiments.

As a result, a significant amount of simulation data has been produced, both as simulation input and output, and this data is often made available to the public via online databases.


\textbf{Importance of visualisation in the physiology}

Biochemical processes are the root of the complex physiological processes that allow living organisms to function.

Because physiology truly spans across multiple scientific fields, it is crucial to communicate scientific advances efficiently between experts with various scientific background.

Moreover, there is also a natural interest from the general audience to understand how their own body function.

Explaining such complex processes however, is usually very challenging without a supporting illustration.

Indeed it would be very challenging for and outsider to comprehend a physiology textbook without any graphical explanations.

Different types of supports exists to improve the understanding of the viewer: printed image (still and non-interactive), web applications (still and interactive ), movies (animated and non-interactive) and interactive installations or games (animated and interactive).

Illustrations, such as the ones made by David Goodsell or Graham Johnson are often used to illustrate books about cellular biology and are highly appreciated by the audience.

Their realization is very laborious, and demand a great deal of both skills and time.

Firstly scientific illustrators must gather knowledge from the scientific literature in order to understand to process that they want to describe.

This is the first step in transforming an idea or concept into something visual.

This task demands a strong understanding of biology from the illustrator since articles that hold important information can be quite critic to non-expert people.

The second task is the realization, which consists of drawing the schematization of the process (give example of a process here).

The work of the illustrator is not to be confused with the work of an artist.

An artist has the freedom to hide a message or an idea behind an abstraction curtain, which may often challenge the non-purists.

On the contrary, the job of an illustrator is to make the communication process as clear as possible in order to democratize scientific concepts.

There are many important aspects to take seriously into account when realizing a quality illustration:

\textbf{Composition} (Camera, Arrangements of objects, occlusion)

\textbf{Lighting} (Depth perception, shading).

\textbf{Colors} (Colors).

\textbf{Story Telling}.

Some illustrators prefer working with pencils and brushes (D.Goodsell) while other prefer using computer-aided design software such as Photoshop (2D) or Maya/Cimema4D (3D).

The use of computer graphics software, however, is becoming increasing popular among illustrators because it facilitates the drawing of 3 dimensional data.

Perspective and lighting effects, for instance, are automatically handled by the software leaving more time to the artist to work on other aspects such as Colors, Composition or Storytelling.

The storytelling is perhaps one of the key element of an illustration as it clearly contains the core to the message that has to be conveyed.

Animated content is very powerful in illustration because it efficiently engages the viewer into the story (stupid comparison of people preferring movies to books).

Due to the popularity of digital effects used in the movie industry over the last decades, 3D animation packages and online training because widely available thus popularizing the use of these tools by scientific illustrators.

Talk about the movies INVIVO from harvard like Ivan in his PDF.

Say that the next step-up in scientific illustration is interaction/real-time graphics.

Talk about the currently existing interactive things (Games, tools, installation, VR/AR).

Talk about the importance of interactivity (Faster content creation, non deterministic story telling).

Talk about the limitation is terms of time needed for asset creation.

Talk about the vision, which is an interactive platform that would utilize the data available to  
















Physiology describes how livings organism work, and it spans across multiple scientific fields.

It is therefore important to be able to communicate scientific advances efficiently between experts with various scientific background.

Additionally, there is also a growing interest from the general audience to understand how their own body function.

Explaining such complex processes is usually very challenging without a supporting illustration.

Although some illustrators are still doing hand-made illustrations, many of them are using tools inspired from movie production to accelerate their work.

Despite that, modeling a single scenario is still very time consuming (gathering data and realization).

Moreover, illustrations are usually static, (still images and movies) and do not offer much interactivity.
















In biology the cost of wet lab experiments is significantly higher than the cost of computer-based simulations.

Such simulation have a great value because they provide important insight about the functioning of a system. 

Over the last years the use of computer-based simulation have significantly increase in biology.





In biology an increasing number of experiments are conducted in silico because of the gain in time and running costs of in vivo experiements.

Simulation are used to formulate hypothesis.
Explain the different simulation methods particle based and kinetic models and how do they differ.

The microscope was invented to observe the results of in vivo experiments

\textbf{A microscope can observe cross sections...}

It both shows what how things look and how things work.

The results of computation often need to be visualized in order to be better understood.

Visualization needs a good framework that would serve the same purpose as the microscope.

Computational biology offers currently offer structural and functional information about organisms such as E.Coli.

By integrating the two aspects together it would be possible to emulate processes that are even not observable as in vivo.

With the certain advantage that the viewer would no longer be bound in either space or time. 



\textbf{Importance of visualisation in the physiology}

Physiology describes how livings organism work, and it spans across multiple scientific fields.

It is therefore important to be able to communicate scientific advances efficiently between experts with various scientific background.

Additionally, there is also a growing interest from the general audience to understand how their own body function.

Explaining such complex processes is usually very challenging without a supporting illustration.

Although some illustrators are still doing hand-made illustrations, many of them are using tools inspired from movie production to accelerate their work.

Despite that, modeling a single scenario is still very time consuming (gathering data and realization).

Moreover, illustrations are usually static, (still images and movies) and do not offer much interactivity.



\textbf{Procedural information available in public databases (pathway) could be used in conjunction with structural information to automatically generate animated scenario that would save a scientific illustrators a lot of time}.

Rather than taking few months to create a single scenario, illustrators could potentially generate a plethora of scenario simply based on simulated models.

Additionally, if the simulation is fast enough it would also be possible to generate a scenario that would not be predefined and be interactively changed by the viewer.

The key in building the tools for the next generation of biological illustrations is interactivity.

On the one hand, interactivity engages the viewer into a more immersive story, not only by moving around but by also by modifying the course of the story by introducing or removing key elements in the system and observe the consequences.

On the one hand introducing interactive illustrations and thus, developing rendering and simulation technique that could run in real-time also means facilitating the work of scientific illustrators 
which are currently using tools designed for offline movie creation and are therefore very slow to compute (ray tracing, large physics simulation).











The heart of this thesis is joining together structural and functional data, and the introduction of interactive techniques for the production of biological illustrations.

We see the introduction of interactive illustration as two fold:

.Gain time in producing illustration (render/animation times)
.Offer a more engaging user experience (multiple scenario exploration VR)

%Dealing with mutliple local scales
The challenge in rendering lies in rendering multiple scales when dealing with atomic level data.

%Dealing with mutliple temporal scales
The challenge in animation lies in how to simultaneously show processes operation at different time scales.

%Dealing with interaction and scenario changes
The challenge in interaction lies in the fact that simulation of spatial data is very slow

%Dealing with occlusion
Large amount of data which need to be rendering when observing data o






%Illustration help to understand but they are time consuming (gathering data, and realization).
%
%Computer graphics programs already help illustrators to accelerate their work (Maya, Cinema4D)
%
%But there is still a lot of work to do.
%
%Constant improvement in game technologies in often applied across field to accelerate the artists job in the movie industry (real-time simulation, real-time rendering, interaction)
%
%This help scientific illustrators too, but often they are using dedicated tools which are not mainstream and therefore only very few process is made for them.
% 
%Structure:
%
%Dealing with Large Amount of Data 
%
%Dealing with Multiple Scales.
%
%Dealing with Stochasticity / Introducing interaction
%
%Dealing with Occlusion.


Part A The models

A pathway is a series of events that describe a physiological phenomenon.

It describes which are the actors of a reactions, it what concentration they are present and at which rate they react.

A multitude of different pathways are available from public databases online (give examples).


It exists several modeling techniques that were developed to study the dynamism of a given process.

Kinetic modeling take into account the concentration and reaction rates and predict the number of element present at a given time via numerical integration (ODE).

Although no spatial information is taken into account the variation rates in concentration is very valuable to the biologists.


Another type of physiological modeling is agent-based modeling.

Agent-based modeling was initially developed for studying the behaviour of simple thinking digital entities (Animal population growth/decay)

In the biological each agent represent a reactant and has a defined set of possible actions which are defined by the pathway.

Unlike with kinetic modeling, the simulation is based on stochastic rather than determinism.

The whole system is mimicking the reaction-diffusion principle, where each elements are diffusing randomly and occasionally reacting when in contact with a potentinal reaction partner.



This type of simulation are designed to closely resemble how the process actually behaves, and therefore it would be ideal to exploit this type of results and to introduce it in our virtual microscope.

\textbf{Microscopes cannot zoom as deep or slow down time but we potentially could}

From visualizing such system come a few challenges, like exploration or across multiple spatial or temporal scales or occlusion management.



The computation of agent-based simulation is very slow, the entire simulation must be precomputed prior to the visualization which can take hours or days.

Unfortunately, no interactive change of scenario is possible with this type of simulation methods.

On the other hands kinetic modeling is very fast to compute but does no provide spatial information.




\textbf{B. The structures}

Our cells are composed principally of proteins, lipids and nucleic acids bathing of a sea of very small reactants.

They all are made of single atoms, and the way atom are assembled together defines their structures.

Structural biologist have made gigantic efforts over the decade to capture protein and fiber structures and to make the results available in public online databases (PDB).


Acquisition method are still unable to precisely capture large structures such are entire organelles and cells (mesoscale gap).

Using simulation methods experts are attempting to reproduce the structure of larger elements based on the observation of the quantities and locations of given proteins.













 

It exists different types of simulation 


Part B the structure
Part C the visualization (occlusion)
Part E spatial navigation 
Part F temporal navigation 
Part D scenario interaction 



\section{Background and Related Work}



\section{Conclusion}

\chapter{Research Publications}


%
%
%
%
%
%Since \LaTeX\ is widely used in academia and industry, there exists a plethora of freely accessible introductions to the language.
%Reading through the guide at \url{https://en.wikibooks.org/wiki/LaTeX} serves as a comprehensive overview for most of the functionality and is highly recommended before starting with a thesis in \LaTeX.
%
%\section{Installation}
%
%A full \LaTeX\ distribution\index{distribution} consists of not only of the binaries that convert the source files to the typeset documents, but also of a wide range of packages and their documentation.
%Depending on the operating system, different implementations are available as shown in Table~\ref{tab:distrib}.
%\textbf{Due to the large amount of packages that are in everyday use and due to their high interdependence, it is paramount to keep the installed distribution\index{distribution} up to date.}
%Otherwise, obscure errors and tedious debugging ensue.
%
%\section{Editors}
%
%A multitude of \TeX\ \glspl{texteditor} are available differing in their editing models, their supported operating systems and their feature sets.
%A comprehensive overview of \glspl{texteditor} can be found at the Wikipedia page  \url{https://en.wikipedia.org/wiki/Comparison_of_TeX_editors}.
%TeXstudio (\url{http://texstudio.sourceforge.net/}) is recommended.
%
%\section{Compilation}
%
%Modern editors usually provide the compilation programs to generate \gls{pdf} documents and for most \LaTeX\ source files, this is sufficient.
%More advanced \LaTeX\ functionality, such as glossaries and bibliographies, needs additional compilation steps, however.
%It is also possible that errors in the compilation process invalidate intermediate files and force subsequent compilation runs to fail.
%It is advisable to delete intermediate files (\verb|.aux|, \verb|.bbl|, etc.), if errors occur and persist.
%All files that are not generated by the user are automatically regenerated.
%To compile the current document, the steps as shown in Table~\ref{tab:compile} have to be taken.