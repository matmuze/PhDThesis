\newacronym{ctan}{CTAN}{Comprehensive TeX Archive Network}
\newacronym{faq}{FAQ}{Frequently Asked Questions}
\newacronym{pdf}{PDF}{Portable Document Format}
\newacronym{svn}{SVN}{Subversion}
\newacronym{wysiwyg}{WYSIWYG}{What You See Is What You Get}

\newglossaryentry{texteditor}
{
	name={editor},
	description={A text editor is a type of program used for editing plain text files.}
}

\chapter{Abstract}


This is where the first chapter begins...


\chapter{Related Publications}

This is where the first chapter begins...

\chapter{Introduction}

\section{Introduction}

%\textbf{Importance of visualisation in the physiology}
%
%Physiology describes how livings organism work, and it spans across multiple scientific fields.
%
%It is therefore important to be able to communicate scientific advances efficiently between experts with various scientific background.
%
%Additionally, there is also a growing interest from the general audience to understand how their own body function.
%
%Explaining such complex processes is usually very challenging without a supporting illustration.
%
%Although some illustrators are still doing hand-made illustrations, many of them are using tools inspired from movie production to accelerate their work.
%
%Despite that, modeling a single scenario is still very time consuming (gathering data and realization).
%
%Moreover, illustrations are usually static, (still images and movies) and do not offer much interactivity.


\textbf{Importance of visualisation in the physiology}

Understanding how we work means understanding a cascade of events from the microscopic to macroscopic levels.

To begin being with this ambitious enterprise it is necessary to starting studying how processes work on the atomic level, the smallest level which we  understand to this day.

Biochemical processes are the root of the complex physiological processes that allow living organisms to function.

Because physiology truly spans across multiple scientific fields, and therefore it is crucial to communicate scientific advances efficiently between experts with various scientific background.

Moreover, there is also a natural interest from the general audience to understand how their own body function.

Explaining such complex processes however, is usually very challenging without a supporting illustration.


Indeed it would be very challenging for and outsider to comprehend a physiology textbook without any graphical explanations.

Different types of supports exists to improve the understanding of the viewer: printed image (still and non-interactive), web applications (still and interactive ), movies (animated and non-interactive) and interactive installations or games (animated and interactive).

\textbf{Show an illustration here and describe what is biochemical process shown.}

Illustrations, such as the ones made by David Goodsell or Graham Johnson are often used to illustrate books about cellular biology and are highly appreciated by the audience.

Their realization is very laborious, and demand a great deal of both skills and time.

Firstly scientific illustrators must gather knowledge from the scientific literature in order to understand to process that they want to describe.

This is the first step in transforming an idea or concept into something visual.

This task demands a strong understanding of biology from the illustrator since articles that hold important information can be quite critic to non-expert people.

The second task is the realization, which consists of drawing the schematization of the process (give example of a process here).

The work of the illustrator is not to be confused with the work of an artist.

An artist has the freedom to hide a message or an idea behind an abstraction curtain, which may often challenge the non-purists.

On the contrary, the job of an illustrator is to make the communication process as clear as possible in order to democratize scientific concepts.

There are many important aspects to take seriously into account when realizing a quality illustration:

\textbf{Composition} (Camera, Arrangements of objects, occlusion)

\textbf{Lighting} (Depth perception, shading).

\textbf{Colors} (Colors).

\textbf{Story Telling}.

Some illustrators prefer working with pencils and brushes (D.Goodsell) while other prefer using computer-aided design software such as Photoshop (2D) or Maya/Cimema4D (3D).

The use of computer graphics software, however, is becoming increasing popular among illustrators because it facilitates the drawing of 3 dimensional data.

Perspective and lighting effects, for instance, are automatically handled by the software leaving more time to the artist to work on other aspects such as Colors, Composition or Storytelling.

The storytelling is a key element of an illustration as it clearly contains the core to the message that has to be conveyed.

Using illustrated animation is a way to engages the viewer into the story and ensure a maximum degree of comprehension.

%Animated content is very powerful in illustration because it efficiently engages the viewer into the story (stupid comparison of people preferring movies to books).

Due to the popularity of digital effects used in the movie industry over the last decades, 3D animation packages and online training became widely available thus popularizing the use of these tools by scientific illustrators.

One notorious animated educational material is "The Inner Life of the Cell" realized by the XVIVO medical illustration studio appointed by Harward University.

The breathtaking video depicts in short consecutive sequences the complex physiological processes a living cell.

Each local events are embedded in their logical surrounding allowing us to understand the correlation between each on these.

The sequences also seamlessly switch from different zoom levels to understand the impact of small reaction on a higher level.

For reaching such standard in term of animation quality it took them 14 months to design, model, animate and render this movie.


Interactivity, as in a computer game, is another way to engage the viewer even deeper into a scenario.

By asking the player to interact with the presented content, the player is carried in the virtual world and his attention in naturally increased. 

Additionally, the system of reward traditionally present in computer games ensure that the motivation of the player to play the game is preserved over time (Game based-learning).

Immune Attack or Sim Cell are two examples of edutainment titles and their goal is to reveal the functioning of living cell through accomplishment of actions that are part of the physiological processes.

Over the last years, the net revenues of the computer game industry have largely outperformed those of the movie industry, which was not the case a few years ago .

This trend clearly indicates that the interest of the public in interactive entertainments is increasing.

We can also observe a trend is recent AAA titles more towards linear and interactive storytelling rather than pure gameplay.

Those titles usually feature cinematic sequences and  interleaved between interactive sequences where the player has to successfully achieve a task.

This type of games allow to develop complicated and fascinating story with while keeping a low level of complexity in term of realization compared a non-linear scenario.

Some of the more recent titles have even manage to provide similar quality story in terms of complexity and animation quality for non-linear scenario.

A non-linear scenario means that the players decision during the game would influence the course of the story, which allows a more unique and personal user experience.

This type of game can usually keep the player engaged for a duration of 50 to 100 hours of gameplay.

Moreover, the recent emergence of a new generation of virtual and augmented reality hardware such as the Occulus, HTC Vive, Microsoft Hololens, are now paving the way to the way towards the next generation of computer games which promise to be more engaging and realistic than even.

Given the raising interest of the general audience in this type of interactive experience we can imagine that this could be hugely effective for edutainment and scientific dissemination of cell biology.

The creation of interactive content, however, may be much more challenging than for the production of movie due to fast framerates needed in computer games (20 to 60 fps even more for VR) and the multitude of scenario that would need to be modeled in the case of a non-linear user experience.

Given that the creation of a 5 to 10 minutes long explanatory video about cell biology may take up to several months or even years it is clear that the next generation of scientific illustrations is going to need efficient tools to facilitate the content creation.

What we need is:
a. Faster tools to render
	Rendering several hundreds of thousands of molecular agents can take up to dozens of minutes to render a single image depending of the image quality presets such as resolution or lighting effects.

b. Faster tools to compose
	Creating large structures requires assembling thousands of proteins or lipids together based on available knowledge present in scientific literature.

c. Faster tools to animate
	Animation of a large amount of proteins, requires using physics simulation tools that similarly to rendering software are very time consuming for large scenes.

d. Faster tools to generate story telling
	A great part to the creation process is spent understanding a given process in order to depict the clearest visual explanation as possible.
	Creating an animated scenario for an entire living organisms, such as the E.Coli bacteria would require spending a lot of time collecting, reading and understanding all the physiological processes that allow the cell to function.




\textbf{Computational Biology to drive story telling}

%In biology the cost of wet lab experiments is significantly higher than the cost of computer-based simulations.
%
%Such simulation have a great value because they provide important insight about the functioning of a system. 
%
%Over the last years the use of computer-based simulation have significantly increase in biology.

In biochemistry, it exists two types of ways to conduct analysis and understand how processes work, called dry and wet laboratories.

Wet laboratories are laboratories where chemical agents are phyiscally manipulated and then observed.

Dry laboratories are laboratories where computational or mathematical methods are employed for the modelling and analysis of biochemical processes. 

Over the last decades, the use of in-silico (dry lab) modeling and analysis has significantly increased due to development new software and the decreasing costs of super computers.

\textit{The goal of a model is to approximate a given process, a complete and accurate model might simply too complex to describe of to simulate, also some aspects may sometimes be unknown}

\textit{Isolate a single part of the entire machinery to easily study its mechanism.}

Because it may often impractical to accurately model a process in its entirety, dry laboratory experiments are often criticized for their lack of precision.

The analysis of such models, although often inaccurate, is nonetheless still quite valuable for researchers.

In 2013 Martin Karplus, Michael Levitt and Arieh Warshel were awarded the Nobel prize in Chemistry for their work of theoretical modelling for complex chemical system.

Their work showcase the importance of theoretical modelling as a tool to complement experimental techniques.

Indeed the collected information is served to formulate new hypothesis which can be later on verified in wet laboratories.

Wet-lab experiments are usually expensive to conduct, and therefore the analysis of theoretical modeling may bring guidance to researchers and save the time and money needed to run too many wet lab experiments.

As a result of the increase in popularity of dry lab experiments, a significant amount of simulation data has been gathered (input) and produced (output).

The data most frequently stored in digital format and can be easily shared to other peers via online databases.

Available data may comprise structural information (how things look) or procedural information (what to they do) for a large number of biochemical agents.

\textbf{What is structure information ?? give concrete example}

\textbf{New mesocale models are being developed to better understand the pysiology of viruses and cells.}
%Right now only static models exists, but thanks to our technology we are paving the way toward interactive visualization of larger dynamic structures.

\textbf{What is procedural information ?? give concrete example}

At this stage they have very limited ways to see how these models of physiology behave. 

Such a visual form is an abstraction that is hard to relate to what is visually observed in experiments. 

In interdisciplinary physiological sciences this might hamper communication of results. 

Based on these two type of information it would be theoretically possible to digitally reproduce a given biochemical process with a certain degree of accuracy.

\textbf{An emulation of a biochemical process which is as accurate but not exact would allow anyone to concretely observe one or several process in action}

\textit{Depending on how intuitively the visualized data would be presented to the viewer, }

This technology could save countless hours of work to illustrators because interesting scenario could be simply queried and fetched from a database and automatically scripted into digital storyline.

Furthermore assuming that the rendering and simulation would be real-time capable, the manipulation of the simulation parameters would provide the means to alter the course of the scenario on-the-fly, thus offering a interactive and engaging experience for the viewer.

\textbf{Injection of new elements leading to a change of state in the organism goes here.}

This might be the technology that would revolutionize how to effectively communicate how these processes work.

\textit{Biology, medicine, and other sciences can strongly profit from a visualization of physiology in order to gain, verify, and communicate the knowledge and the hypotheses in this field.}

\textbf{Microscope analogy ??}

\textbf{Mention analogy between the microscope as a tool for viewing experimental data, but the lack of unified tool for viewing theoretical data.}

\textbf{Ant-farm analogy ??}



%In biology an increasing number of experiments are conducted in silico because of the gain in time and running costs of in vivo experiements.
%
%Simulation are used to formulate hypothesis.
%Explain the different simulation methods particle based and kinetic models and how do they differ.
%
%The microscope was invented to observe the results of in vivo experiments
%
%\textbf{A microscope can observe cross sections...}
%
%It both shows what how things look and how things work.
%
%The results of computation often need to be visualized in order to be better understood.
%
%Visualization needs a good framework that would serve the same purpose as the microscope.
%
%Computational biology offers currently offer structural and functional information about organisms such as E.Coli.
%
%By integrating the two aspects together it would be possible to emulate processes that are even not observable as in vivo.
%
%With the certain advantage that the viewer would no longer be bound in either space or time. 

%
%\textbf{Procedural information available in public databases (pathway) could be used in conjunction with structural information to automatically generate animated scenario that would save a scientific illustrators a lot of time}.
%
%Rather than taking few months to create a single scenario, illustrators could potentially generate a plethora of scenario simply based on simulated models.
%
%Additionally, if the simulation is fast enough it would also be possible to generate a scenario that would not be predefined and be interactively changed by the viewer.
%
%The key in building the tools for the next generation of biological illustrations is interactivity.
%
%On the one hand, interactivity engages the viewer into a more immersive story, not only by moving around but by also by modifying the course of the story by introducing or removing key elements in the system and observe the consequences.
%
%On the one hand introducing interactive illustrations and thus, developing rendering and simulation technique that could run in real-time also means facilitating the work of scientific illustrators 
%which are currently using tools designed for offline movie creation and are therefore very slow to compute (ray tracing, large physics simulation).


%***************************************************************

%Say that the next step-up in scientific illustration is interaction/real-time graphics.
%
%Talk about the currently existing interactive things (Games, tools, installation, VR/AR).
%
%Talk about the importance of interactivity (Faster content creation, non deterministic story telling).
%
%Talk about the limitation is terms of time needed for asset creation.
%
%Talk about the vision, which is an interactive platform that would utilize the data available to  

%***************************************************************

\textbf{Towards a dynamic and interactive model of a cell (E.Coli)}

\textbf{Requirement analysis}

The heart of this thesis is joining together structural and functional data, and the introduction of interactive techniques for the production of biological illustrations.

We see the introduction of interactive illustration as two fold:

.Gain time in producing illustration (render/animation times)
.Offer a more engaging user experience (multiple scenario exploration VR)

%Dealing with mutliple local scales
The challenge in rendering lies in rendering multiple scales when dealing with atomic level data.

%Dealing with mutliple temporal scales
The challenge in animation lies in how to simultaneously show processes operation at different time scales.

%Dealing with interaction and scenario changes
The challenge in interaction lies in the fact that simulation of spatial data is very slow

%Dealing with occlusion
Large amount of data which need to be rendering when observing data o


%Illustration help to understand but they are time consuming (gathering data, and realization).
%
%Computer graphics programs already help illustrators to accelerate their work (Maya, Cinema4D)
%
%But there is still a lot of work to do.
%
%Constant improvement in game technologies in often applied across field to accelerate the artists job in the movie industry (real-time simulation, real-time rendering, interaction)
%
%This help scientific illustrators too, but often they are using dedicated tools which are not mainstream and therefore only very few process is made for them.
% 
%Structure:
%
%Dealing with Large Amount of Data 
%
%Dealing with Multiple Scales.
%
%Dealing with Stochasticity / Introducing interaction
%
%Dealing with Occlusion.


Part A The models

A pathway is a series of events that describe a physiological phenomenon.

It describes which are the actors of a reactions, it what concentration they are present and at which rate they react.

A multitude of different pathways are available from public databases online (give examples).


It exists several modeling techniques that were developed to study the dynamism of a given process.

Kinetic modeling take into account the concentration and reaction rates and predict the number of element present at a given time via numerical integration (ODE).

Although no spatial information is taken into account the variation rates in concentration is very valuable to the biologists.


Another type of physiological modeling is agent-based modeling.

Agent-based modeling was initially developed for studying the behaviour of simple thinking digital entities (Animal population growth/decay)

In the biological each agent represent a reactant and has a defined set of possible actions which are defined by the pathway.

Unlike with kinetic modeling, the simulation is based on stochastic rather than determinism.

The whole system is mimicking the reaction-diffusion principle, where each elements are diffusing randomly and occasionally reacting when in contact with a potentinal reaction partner.



This type of simulation are designed to closely resemble how the process actually behaves, and therefore it would be ideal to exploit this type of results and to introduce it in our virtual microscope.

\textbf{Microscopes cannot zoom as deep or slow down time but we potentially could}

From visualizing such system come a few challenges, like exploration or across multiple spatial or temporal scales or occlusion management.



The computation of agent-based simulation is very slow, the entire simulation must be precomputed prior to the visualization which can take hours or days.

Unfortunately, no interactive change of scenario is possible with this type of simulation methods.

On the other hands kinetic modeling is very fast to compute but does no provide spatial information.


\textbf{B. The structures}

Our cells are composed principally of proteins, lipids and nucleic acids bathing of a sea of very small reactants.

They all are made of single atoms, and the way atom are assembled together defines their structures.

Structural biologist have made gigantic efforts over the decade to capture protein and fiber structures and to make the results available in public online databases (PDB).


Acquisition method are still unable to precisely capture large structures such are entire organelles and cells (mesoscale gap).

Using simulation methods experts are attempting to reproduce the structure of larger elements based on the observation of the quantities and locations of given proteins.
 

It exists different types of simulation 


Part B the structure
Part C the visualization (occlusion)
Part E spatial navigation 
Part F temporal navigation 
Part D scenario interaction 



\section{Background and Related Work}



\section{Conclusion}

\chapter{Research Publications}


%
%
%
%
%
%Since \LaTeX\ is widely used in academia and industry, there exists a plethora of freely accessible introductions to the language.
%Reading through the guide at \url{https://en.wikibooks.org/wiki/LaTeX} serves as a comprehensive overview for most of the functionality and is highly recommended before starting with a thesis in \LaTeX.
%
%\section{Installation}
%
%A full \LaTeX\ distribution\index{distribution} consists of not only of the binaries that convert the source files to the typeset documents, but also of a wide range of packages and their documentation.
%Depending on the operating system, different implementations are available as shown in Table~\ref{tab:distrib}.
%\textbf{Due to the large amount of packages that are in everyday use and due to their high interdependence, it is paramount to keep the installed distribution\index{distribution} up to date.}
%Otherwise, obscure errors and tedious debugging ensue.
%
%\section{Editors}
%
%A multitude of \TeX\ \glspl{texteditor} are available differing in their editing models, their supported operating systems and their feature sets.
%A comprehensive overview of \glspl{texteditor} can be found at the Wikipedia page  \url{https://en.wikipedia.org/wiki/Comparison_of_TeX_editors}.
%TeXstudio (\url{http://texstudio.sourceforge.net/}) is recommended.
%
%\section{Compilation}
%
%Modern editors usually provide the compilation programs to generate \gls{pdf} documents and for most \LaTeX\ source files, this is sufficient.
%More advanced \LaTeX\ functionality, such as glossaries and bibliographies, needs additional compilation steps, however.
%It is also possible that errors in the compilation process invalidate intermediate files and force subsequent compilation runs to fail.
%It is advisable to delete intermediate files (\verb|.aux|, \verb|.bbl|, etc.), if errors occur and persist.
%All files that are not generated by the user are automatically regenerated.
%To compile the current document, the steps as shown in Table~\ref{tab:compile} have to be taken.