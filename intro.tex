\newacronym{ctan}{CTAN}{Comprehensive TeX Archive Network}
\newacronym{faq}{FAQ}{Frequently Asked Questions}
\newacronym{pdf}{PDF}{Portable Document Format}
\newacronym{svn}{SVN}{Subversion}
\newacronym{wysiwyg}{WYSIWYG}{What You See Is What You Get}

\newglossaryentry{texteditor}
{
	name={editor},
	description={A text editor is a type of program used for editing plain text files.}
}

\chapter{Abstract}


This is where the first chapter begins...


\chapter{Related Publications}


This is where the first chapter begins...

\chapter{Overview}


This is where the first chapter begins...

\section{Introduction}

\section{Background and Related Work}

\section{Background and Related Work}

\section{Conclusion}







\chapter{Research Publications}


%
%
%
%
%
%Since \LaTeX\ is widely used in academia and industry, there exists a plethora of freely accessible introductions to the language.
%Reading through the guide at \url{https://en.wikibooks.org/wiki/LaTeX} serves as a comprehensive overview for most of the functionality and is highly recommended before starting with a thesis in \LaTeX.
%
%\section{Installation}
%
%A full \LaTeX\ distribution\index{distribution} consists of not only of the binaries that convert the source files to the typeset documents, but also of a wide range of packages and their documentation.
%Depending on the operating system, different implementations are available as shown in Table~\ref{tab:distrib}.
%\textbf{Due to the large amount of packages that are in everyday use and due to their high interdependence, it is paramount to keep the installed distribution\index{distribution} up to date.}
%Otherwise, obscure errors and tedious debugging ensue.
%
%\section{Editors}
%
%A multitude of \TeX\ \glspl{texteditor} are available differing in their editing models, their supported operating systems and their feature sets.
%A comprehensive overview of \glspl{texteditor} can be found at the Wikipedia page  \url{https://en.wikipedia.org/wiki/Comparison_of_TeX_editors}.
%TeXstudio (\url{http://texstudio.sourceforge.net/}) is recommended.
%
%\section{Compilation}
%
%Modern editors usually provide the compilation programs to generate \gls{pdf} documents and for most \LaTeX\ source files, this is sufficient.
%More advanced \LaTeX\ functionality, such as glossaries and bibliographies, needs additional compilation steps, however.
%It is also possible that errors in the compilation process invalidate intermediate files and force subsequent compilation runs to fail.
%It is advisable to delete intermediate files (\verb|.aux|, \verb|.bbl|, etc.), if errors occur and persist.
%All files that are not generated by the user are automatically regenerated.
%To compile the current document, the steps as shown in Table~\ref{tab:compile} have to be taken.