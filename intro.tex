\newacronym{ctan}{CTAN}{Comprehensive TeX Archive Network}
\newacronym{faq}{FAQ}{Frequently Asked Questions}
\newacronym{pdf}{PDF}{Portable Document Format}
\newacronym{svn}{SVN}{Subversion}
\newacronym{wysiwyg}{WYSIWYG}{What You See Is What You Get}

\newglossaryentry{texteditor}
{
	name={editor},
	description={A text editor is a type of program used for editing plain text files.}
}

\chapter{Abstract}


This is where the first chapter begins...


\chapter{Related Publications}


This is where the first chapter begins...

\chapter{Introduction}


This is where the first chapter begins...

\section{Foreword}

From here it is important to show how the multiple elements of the thesis blend together.

Multiple zoom levels, multiple temporal levels

Towards a dynamic and interactive model of a cell (E.Coli)

Simulation are used to formulate hypothesis.
Explain the different simulation methods particle based and kinetic models and how do they differ.

New mesocale models are being developed to better understand the pysiology of viruses and cells.
Right now only static models exists, but thanks to our technology we are paving the way toward interactive visualization of larger dynamic structures.



%Understanding how we work means understanding a cascade of events from the microscopic to macroscopic levels.
%
%To begin being with this ambitious enterprise it is necessary to starting studying how processes work on the atomic level, the smallest level which we  understand to this day.

Physiology describes how livings organism work, and it spans across multiple scientific fields.

It is therefore important to be able to communicate scientific advances efficiently between experts with various scientific background.

Additionally, there is also a growing interest from the general audience to understand how their own body function.

Explaining such complex processes is usually very challenging without a supporting illustration.

Although some illustrators are still doing hand-made illustrations, many of them are using tools inspired from movie production to accelerate their work.

Despite that, modeling a single scenario is still very time consuming (gathering data and realization).

Moreover, illustrations are usually static, (still images and movies) and do not offer much interactivity.

The heart of this thesis is in the introduction of interactive techniques for the production of biological illustrations.

We see the introduction of interactive illustration as two fold:

.Gain time in producing illustration (render/animation times)
.Offer a more engaging user experience (multiple scenario exploration VR)

%Dealing with mutliple local scales
The challenge in rendering lies in rendering multiple scales when dealing with atomic level data.



%Dealing with mutliple temporal scales
The challenge in animation lies in how to simultaneously show processes operation at different time scales.

%Dealing with interaction and scenario changes
The challenge in interaction lies in the fact that simulation of spatial data is very slow

%Dealing with occlusion
Large amount of data which need to be rendering when observing data o



%Illustration help to understand but they are time consuming (gathering data, and realization).
%
%Computer graphics programs already help illustrators to accelerate their work (Maya, Cinema4D)
%
%But there is still a lot of work to do.
%
%Constant improvement in game technologies in often applied across field to accelerate the artists job in the movie industry (real-time simulation, real-time rendering, interaction)
%
%This help scientific illustrators too, but often they are using dedicated tools which are not mainstream and therefore only very few process is made for them.
% 
%Structure:
%
%Dealing with Large Amount of Data 
%
%Dealing with Multiple Scales.
%
%Dealing with Stochasticity / Introducing interaction
%
%Dealing with Occlusion.




\section{Background and Related Work}



\section{Conclusion}

\chapter{Research Publications}


%
%
%
%
%
%Since \LaTeX\ is widely used in academia and industry, there exists a plethora of freely accessible introductions to the language.
%Reading through the guide at \url{https://en.wikibooks.org/wiki/LaTeX} serves as a comprehensive overview for most of the functionality and is highly recommended before starting with a thesis in \LaTeX.
%
%\section{Installation}
%
%A full \LaTeX\ distribution\index{distribution} consists of not only of the binaries that convert the source files to the typeset documents, but also of a wide range of packages and their documentation.
%Depending on the operating system, different implementations are available as shown in Table~\ref{tab:distrib}.
%\textbf{Due to the large amount of packages that are in everyday use and due to their high interdependence, it is paramount to keep the installed distribution\index{distribution} up to date.}
%Otherwise, obscure errors and tedious debugging ensue.
%
%\section{Editors}
%
%A multitude of \TeX\ \glspl{texteditor} are available differing in their editing models, their supported operating systems and their feature sets.
%A comprehensive overview of \glspl{texteditor} can be found at the Wikipedia page  \url{https://en.wikipedia.org/wiki/Comparison_of_TeX_editors}.
%TeXstudio (\url{http://texstudio.sourceforge.net/}) is recommended.
%
%\section{Compilation}
%
%Modern editors usually provide the compilation programs to generate \gls{pdf} documents and for most \LaTeX\ source files, this is sufficient.
%More advanced \LaTeX\ functionality, such as glossaries and bibliographies, needs additional compilation steps, however.
%It is also possible that errors in the compilation process invalidate intermediate files and force subsequent compilation runs to fail.
%It is advisable to delete intermediate files (\verb|.aux|, \verb|.bbl|, etc.), if errors occur and persist.
%All files that are not generated by the user are automatically regenerated.
%To compile the current document, the steps as shown in Table~\ref{tab:compile} have to be taken.