\newacronym{ctan}{CTAN}{Comprehensive TeX Archive Network}
\newacronym{faq}{FAQ}{Frequently Asked Questions}
\newacronym{pdf}{PDF}{Portable Document Format}
\newacronym{svn}{SVN}{Subversion}
\newacronym{wysiwyg}{WYSIWYG}{What You See Is What You Get}

\newglossaryentry{texteditor}
{
	name={editor},
	description={A text editor is a type of program used for editing plain text files.}
}

\chapter{Abstract}


This is where the first chapter begins...


\chapter{Related Publications}

This is where the first chapter begins...

\chapter{Introduction}

\section{Introduction}

%\textbf{Importance of visualisation in the physiology}
%
%Physiology describes how livings organism work, and it spans across multiple scientific fields.
%
%It is therefore important to be able to communicate scientific advances efficiently between experts with various scientific background.
%
%Additionally, there is also a growing interest from the general audience to understand how their own body function.
%
%Explaining such complex processes is usually very challenging without a supporting illustration.
%
%Although some illustrators are still doing hand-made illustrations, many of them are using tools inspired from movie production to accelerate their work.
%
%Despite that, modeling a single scenario is still very time consuming (gathering data and realization).
%
%Moreover, illustrations are usually static, (still images and movies) and do not offer much interactivity.


\textbf{Importance of Visualisation in Biology}


%To being with this ambitious enterprise it is necessary to starting studying how processes work on the atomic level, the smallest level which we  understand to this day.
Biochemistry is the root of complex physiological processes that allow living organisms to function.
In order to understanding how we work we must understand the complete cascade of events from the macroscopic down to microscopic and atomic levels.
Because physiology truly spans across multiple scientific fields, it is crucial to communicate advances in biochemistry efficiently between experts with various scientific background.
Moreover, there is also a growing interest from the general audience to understand how their own biological machinery works.
Visual communication is without a doubt a very efficient way to educate a non-expert audience about the functioning of physiological processes.
A quick glance at physiological textbooks is enough to realize that they would be close to useless without any expression illustrations.
Printed illustrations have the clear advantage to be easily accessible by the viewer, via a school textooks, scientific magazines, encyclopedia, or simply via the web. 
Illustrations, such as the ones made by David Goodsell, for example, are often used to illustrate cellular biology textbooks.
The illustrator like to depict entire sceneries on the mesoscale levels that would be impossible to observe otherwise in such detail  using current optical instruments.
His illustrations have acquired a certain notoriety due their quality which can be assessed by two factors:
\\
\\ \textbf{How close is the depiction to the reality ?}
\\ \textbf{How clear is the depiction to the viewer ?}
\textbf{Show an illustration here and describe what is biochemical process shown.}
The realization of such illustrations, however, is very laborious and demand a lot of skills and time.
The first step is the conception, which consists of gathering knowledge from the literature in order to thoroughly understand the process that should be depicted.
%This step is crucial as it will determine knowledge to picture translation process.
Based on this knowledge the illustrator will decide which elements should be present, their location, quantities, and which behaviour should they exhibit.
This task demands a strong understanding of biology as scientific articles are intended to be read by experts and peers.
The second step is the realization, which consists of drawing a systematization of the process (give example of a process here).
It is important not to confuse the work of a scientific illustrator and the work of an artist.
Although they both aim at conveying a message or an idea, the artist has the freedom to hide him message behind an abstraction curtain in order to challenge the viewer.
On the contrary, the job of an illustrator is to make the communication process as clear as possible in order to expose scientific concepts to a laymen audience.
There are several key aspects to seriously take into account in order to achieve a high quality illustration:
\\
\\ \textbf{Composition} (Camera, Arrangements of objects, occlusion)
\\ \textbf{Lighting} (Depth perception, shading).
\\ \textbf{Color Coding} (Colors).
\\ \textbf{Storytelling}.

It also exists different types of supports other than still images such as animated movies (animated and non-interactive) and educational games (animated and interactive).
Some illustrators prefer working with pencils and brushes (D.Goodsell) while other prefer using computer-aided design software such as Photoshop (2D) or Maya/Cimema4D (3D).
The use of computer graphics software, however, is becoming increasing popular among illustrators because it greatly facilitates the drawing of 3D dimensional data, which is also becoming the standard in animated movie creation.
Perspective and lighting effects, for instance, are automatically handled by the software leaving more time to the artist to work on other aspects such as color coding, composition, animation or storytelling.
The storytelling is a key element of an illustration as it is responsible of conveying the core message.

Animation is a great asset for storytelling as it is great way to engage the viewer into a story and to ensure a maximum degree of comprehension.
%Animated content is very powerful in illustration because it efficiently engages the viewer into the story (stupid comparison of people preferring movies to books).
Due to the popularity of digital effects used in the movie industry over the last decades, 3D animation packages and online training became widely available thus popularizing the use of these tools by scientific illustrators.
One notorious animated educational material is "The Inner Life of the Cell" realized in 2008 by the XVIVO medical illustration studio and appointed by Harward University.
This short animated video beautifully depicts in few sequences the basics of the complex biochemical processes of a living cell such as.....
\textbf{An immense effort was spend to only to show local reaction but also to represent the surroundings faithfully}
The local molecular interactions are embedded in their logical surrounding, which allows the viewer to understand the correlation when transitioning to a subsequent interaction.
The sequences also seamlessly transit between different zoom levels to understand the impact of small reaction on a higher level.
For reaching such standard in term of animation quality it took them 14 months to design, model, animate and render this movie.

Another way to engage the subject even deeper into a scenario is interactivity.
By embedding the subject into the presented content, he becomes an actor of the virtual world and his attention naturally increases. 
Moreover, the traditional reward system present in most computer games ensures that the player stay engaged for a very long duration (Game based-learning).
The motivation of a player, however, is not always solely driven by the high-scores.
We can observe a recent trend among AAA game titles towards interactive storytelling rather than pure gameplay.
Those titles usually feature cinematic sequences, interleaved between interactive sessions where the player has to successfully achieve a task.
The player motivation is then also driven by his curiosity to unravel the storyline.
This type of game usually keeps the player engaged for a duration of 50 to 100 hours.
Over the last years, the net revenues of the computer game industry have largely outperformed those of the movie industry, which was not the case a few years ago.
This trend clearly indicates the growing interest of the public in interactive entertainment and  storytelling. 
Moreover, the recent emergence of a new generation of virtual and augmented reality hardware such as the Occulus, HTC Vive, Microsoft Hololens, are now paving the way towards the next generation of computer games which promise to be even more immersive, realistic and engaging.

Given the raising interest of the general audience in this type of interactive experience we can imagine that this could be hugely effective for edutainment and scientific dissemination of cell biology.
Immune Attack or Sim Cell are two famous examples of edutainment titles whose goal is to reveal the functioning of living cell through accomplishment of actions that are part of the physiological processes.
\textbf{Those a nice examples but still quite low buget, too few titles out there}
The realization of interactive content, however, is more constrained compared to production of movies which can be a real challenge. Indeed due to fast framerates needed in computer games (20 to 60 fps even more for VR) it difficult to reproduce entire complex environment we may see in movies, which may affect the overall communication impact. 
Additionally the creation team also requires more people e.g. computer programmers, which may increase the production costs and times. 
Given that the creation of a high quality explanatory video about cell biology may already take up to several months or even years it is clear that the next generation of interactive scientific illustrations is going to need efficient tools to facilitate the content creation.

What we need is:
a. Faster tools to render
	Rendering several hundreds of thousands of molecular agents can take up to dozens of minutes to render a single image depending of the image quality presets such as resolution or lighting effects.

b. Faster tools to compose
	Creating large structures requires assembling thousands of proteins or lipids together based on available knowledge present in scientific literature.

c. Faster tools to generate story telling
	A great part to the creation process is spent understanding a given process in order to depict the clearest visual explanation as possible.
	Creating an animated scenario for an entire living organisms, such as the E.Coli bacteria would require spending a lot of time collecting, reading and understanding all the physiological processes that allow the cell to function.

%c. Faster tools to animate
%	Animation of a large amount of proteins, requires using physics simulation tools that similarly to rendering software are very time consuming for large scenes.

\section{Background}

\textbf{Computational Biology to Assist Storytelling}

%In biology the cost of wet lab experiments is significantly higher than the cost of computer-based simulations.
%
%Such simulation have a great value because they provide important insight about the functioning of a system. 
%
%Over the last years the use of computer-based simulation have significantly increase in biology.

In biochemistry, it exists two methods to conduct analysis and understand how processes work, called dry and wet laboratories.
Wet laboratories are laboratories where chemical agents are phyiscally manipulated and then observed.
Dry laboratories are laboratories where computational or mathematical methods are employed for the modelling and analysis of biochemical processes. 
Over the last decades, the use of in-silico (dry lab) modeling and analysis has significantly increased due to development new software and the decreasing costs of super computers.
\textit{The goal of a model is to approximate a given process, a complete and accurate model might simply too complex to describe of to simulate, also some aspects may sometimes be unknown}
\textit{Isolate a single part of the entire machinery to easily study its mechanism.}
Because it may often impractical to accurately model a process in its entirety, dry laboratory experiments are often criticized for their lack of precision.
The analysis of such models, although often inaccurate, is nonetheless still quite valuable for researchers.
In 2013 Martin Karplus, Michael Levitt and Arieh Warshel were awarded the Nobel prize in Chemistry for their work of theoretical modelling for complex chemical system.
Their work showcase the importance of theoretical modelling as a tool to complement experimental techniques.
Indeed the collected information is served to formulate new hypothesis which can be later on verified in wet laboratories.
Wet-lab experiments are usually expensive to conduct, and therefore the analysis of theoretical modeling may bring guidance to researchers and save the time and money needed to run too many wet lab experiments.
As a result of the increase in popularity of dry lab experiments, a significant amount of simulation data has been gathered (input) and produced (output).
The data most frequently stored in digital format and can be easily shared to other peers via online databases.
Available data may comprise structural information (how things look) or procedural information (what to they do) for a large number of biochemical agents.
\textbf{What is structure information ?? give concrete example}
\textbf{New mesocale models are being developed to better understand the pysiology of viruses and cells.}
%Right now only static models exists, but thanks to our technology we are paving the way toward interactive visualization of larger dynamic structures.
\textbf{What is procedural information ?? give concrete example}
At this stage they have very limited ways to see how these models of physiology behave. 
Such a visual form is an abstraction that is hard to relate to what is visually observed in experiments. 
In interdisciplinary physiological sciences this might hamper communication of results. 
Based on these two information, however, it would be theoretically possible to digitally produce a visualization that would illustrate a given biochemical process.
This technology could save countless hours of work to illustrators because interesting scenario could be simply queried and fetched from a database and automatically scripted into digital storyline.
Furthermore assuming that the rendering and simulation to be real-time capable, the manipulation of the simulation parameters would provide the means to alter the course of the scenario on-the-fly, thus offering a interactive and engaging experience for the viewer.
\textbf{Add a concrete example to physiological process which could be interesting to interact with}.
This might be the technology that would revolutionize how to effectively communicate how these processes work.
\textit{Biology, medicine, and other sciences can strongly profit from a visualization of physiology in order to gain, verify, and communicate the knowledge and the hypotheses in this field.}
\textbf{A common framework to bridge real-time simulation and meaningful 3D visualization together}

%\textbf{Microscope analogy ??}
%
%\textbf{Mention analogy between the microscope as a tool for viewing experimental data, but the lack of unified tool for viewing theoretical data.}
%
%\textbf{Ant-farm analogy ??}

\subsection{Biological Systems}

\textbf{What is a process ? how to model it ?}

The goal of a model is to describe an a complete or partial \textit{biochemical} biological process in order to simulate and analyse its functioning in more details.
The scientist who simulate such models interested in the properties of certain species, such as concentration, reaction rates and sometimes their spatial distribution.
Most commonly they use computer-based simulation programs to gauge how those properties change over time.
Based on this information and their own knowledge they are able to formulate hypothesis which helps them to broaden their understanding of the machinery of life. 
The models, as we understand them, describe the different interaction between species, which elements react with each other, what is the product of the reaction, and how fast do they react.
Additionally the concentration of each individual species defines the initial state of the simulation.
It is common to describe these interactions as a reaction network, called pathway, to visually communicate given processes in biology books for instance.
A complete description of the reaction network can also be represented with a custom markup language (such as SBML) and then stored in a digital file and easily shared with other researchers.
Digital models descriptions are also widely supported by simulation programs to facilitate the initialization procedure.

The computational method employed by the simulation program can either be deterministic or non-deterministic...

%It exists several modeling techniques that were developed to study the dynamism of a given process.
%
%Kinetic modeling take into account the concentration and reaction rates and predict the number of element present at a given time via numerical integration (ODE).
%
%Although no spatial information is taken into account the variation rates in concentration is very valuable to the biologists.
%
%
%Another type of physiological modeling is agent-based modeling.
%
%Agent-based modeling was initially developed for studying the behaviour of simple thinking digital entities (Animal population growth/decay)
%
%In the biological each agent represent a reactant and has a defined set of possible actions which are defined by the pathway.
%
%Unlike with kinetic modeling, the simulation is based on stochastic rather than determinism.
%
%The whole system is mimicking the reaction-diffusion principle, where each elements are diffusing randomly and occasionally reacting when in contact with a potentinal reaction partner.
%
%This type of simulation are designed to closely resemble how the process actually behaves, and therefore it would be ideal to exploit this type of results and to introduce it in our virtual microscope.
%
%\textbf{Microscopes cannot zoom as deep or slow down time but we potentially could}
%
%From visualizing such system come a few challenges, like exploration or across multiple spatial or temporal scales or occlusion management.
%
%
%
%The computation of agent-based simulation is very slow, the entire simulation must be precomputed prior to the visualization which can take hours or days.
%
%Unfortunately, no interactive change of scenario is possible with this type of simulation methods.
%
%On the other hands kinetic modeling is very fast to compute but does no provide spatial information.

\subsection{Biological Structures}

Our cells are composed principally of proteins, lipids and nucleic acids bathing of a sea of very small reactants.
They all are made of single atoms, and the way atom are assembled together defines their structures.
Structural biologist have made gigantic efforts over the decade to capture protein and fiber structures and to make the results available in public online databases (PDB).
Acquisition method are still unable to precisely capture large structures such are entire organelles and cells (mesoscale gap).
Using simulation methods experts are attempting to reproduce the structure of larger elements based on the observation of the quantities and locations of given proteins.

%In biology an increasing number of experiments are conducted in silico because of the gain in time and running costs of in vivo experiements.
%
%Simulation are used to formulate hypothesis.
%Explain the different simulation methods particle based and kinetic models and how do they differ.
%
%The microscope was invented to observe the results of in vivo experiments
%
%\textbf{A microscope can observe cross sections...}
%
%It both shows what how things look and how things work.
%
%The results of computation often need to be visualized in order to be better understood.
%
%Visualization needs a good framework that would serve the same purpose as the microscope.
%
%Computational biology offers currently offer structural and functional information about organisms such as E.Coli.
%
%By integrating the two aspects together it would be possible to emulate processes that are even not observable as in vivo.
%
%With the certain advantage that the viewer would no longer be bound in either space or time. 

%
%\textbf{Procedural information available in public databases (pathway) could be used in conjunction with structural information to automatically generate animated scenario that would save a scientific illustrators a lot of time}.
%
%Rather than taking few months to create a single scenario, illustrators could potentially generate a plethora of scenario simply based on simulated models.
%
%Additionally, if the simulation is fast enough it would also be possible to generate a scenario that would not be predefined and be interactively changed by the viewer.
%
%The key in building the tools for the next generation of biological illustrations is interactivity.
%
%On the one hand, interactivity engages the viewer into a more immersive story, not only by moving around but by also by modifying the course of the story by introducing or removing key elements in the system and observe the consequences.
%
%On the one hand introducing interactive illustrations and thus, developing rendering and simulation technique that could run in real-time also means facilitating the work of scientific illustrators 
%which are currently using tools designed for offline movie creation and are therefore very slow to compute (ray tracing, large physics simulation).


%***************************************************************

%Say that the next step-up in scientific illustration is interaction/real-time graphics.
%
%Talk about the currently existing interactive things (Games, tools, installation, VR/AR).
%
%Talk about the importance of interactivity (Faster content creation, non deterministic story telling).
%
%Talk about the limitation is terms of time needed for asset creation.
%
%Talk about the vision, which is an interactive platform that would utilize the data available to  

%***************************************************************


%-----------------------------------------------

%\textbf{Towards a dynamic and interactive model of a cell (E.Coli)}
%
%\textbf{Requirement analysis}
%
%The heart of this thesis is joining together structural and functional data, and the introduction of interactive techniques for the production of biological illustrations.
%
%We see the introduction of interactive illustration as two fold:
%
%.Gain time in producing illustration (render/animation times)
%.Offer a more engaging user experience (multiple scenario exploration VR)
%
%%Dealing with mutliple local scales
%The challenge in rendering lies in rendering multiple scales when dealing with atomic level data.
%
%%Dealing with mutliple temporal scales
%The challenge in animation lies in how to simultaneously show processes operation at different time scales.
%
%%Dealing with interaction and scenario changes
%The challenge in interaction lies in the fact that simulation of spatial data is very slow
%
%%Dealing with occlusion
%Large amount of data which need to be rendering when observing data o

%-----------------------------------------------

%Illustration help to understand but they are time consuming (gathering data, and realization).
%
%Computer graphics programs already help illustrators to accelerate their work (Maya, Cinema4D)
%
%But there is still a lot of work to do.
%
%Constant improvement in game technologies in often applied across field to accelerate the artists job in the movie industry (real-time simulation, real-time rendering, interaction)
%
%This help scientific illustrators too, but often they are using dedicated tools which are not mainstream and therefore only very few process is made for them.
% 
%Structure:
%
%Dealing with Large Amount of Data 
%
%Dealing with Multiple Scales.
%
%Dealing with Stochasticity / Introducing interaction
%
%Dealing with Occlusion.

%Part C the visualization (occlusion)
%Part E spatial navigation 
%Part F temporal navigation 
%Part D scenario interaction 

\section{Using Computation Data for Storytelling}

\section{Efficient Rendering of Large Structures}

\section{Efficient Spatial Composition and Occlusion Management}




\section{Conclusion}

\chapter{Research Publications}


%
%
%
%
%
%Since \LaTeX\ is widely used in academia and industry, there exists a plethora of freely accessible introductions to the language.
%Reading through the guide at \url{https://en.wikibooks.org/wiki/LaTeX} serves as a comprehensive overview for most of the functionality and is highly recommended before starting with a thesis in \LaTeX.
%
%\section{Installation}
%
%A full \LaTeX\ distribution\index{distribution} consists of not only of the binaries that convert the source files to the typeset documents, but also of a wide range of packages and their documentation.
%Depending on the operating system, different implementations are available as shown in Table~\ref{tab:distrib}.
%\textbf{Due to the large amount of packages that are in everyday use and due to their high interdependence, it is paramount to keep the installed distribution\index{distribution} up to date.}
%Otherwise, obscure errors and tedious debugging ensue.
%
%\section{Editors}
%
%A multitude of \TeX\ \glspl{texteditor} are available differing in their editing models, their supported operating systems and their feature sets.
%A comprehensive overview of \glspl{texteditor} can be found at the Wikipedia page  \url{https://en.wikipedia.org/wiki/Comparison_of_TeX_editors}.
%TeXstudio (\url{http://texstudio.sourceforge.net/}) is recommended.
%
%\section{Compilation}
%
%Modern editors usually provide the compilation programs to generate \gls{pdf} documents and for most \LaTeX\ source files, this is sufficient.
%More advanced \LaTeX\ functionality, such as glossaries and bibliographies, needs additional compilation steps, however.
%It is also possible that errors in the compilation process invalidate intermediate files and force subsequent compilation runs to fail.
%It is advisable to delete intermediate files (\verb|.aux|, \verb|.bbl|, etc.), if errors occur and persist.
%All files that are not generated by the user are automatically regenerated.
%To compile the current document, the steps as shown in Table~\ref{tab:compile} have to be taken.